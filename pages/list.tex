\begin{itemize}
\item \textbf{Age}, as in: infant, child, adolescent, adult, etc. 
\item \textbf{Fat/thin}, as cultural labels pertaining to body type. 
\item \textbf{Class/caste}, as in socioeconomic status or caste membership. 
\item \textbf{Disability and/or health condition}, coded separately from conditions considered cognitive disabilities.
\item \textbf{Ethnicity}, broadly conceived in the anthropological sense of deriving group identity from common traditions, culture, society, heritage, etc., but coded separately from ethnic identity constructed via religion or nationality.
\item \textbf{Gender}, including gender identity, performance, and expression.
\item \textbf{Mind-body-identity mapping}, as in the quantity, location, and temporality of mind, body, and identity relative to each other.
\item \textbf{Nationality}, or national origin. 
\item \textbf{Neurodiversity}, such as being on the autism spectrum or experiencing variable attention to stimulus.
\item \textbf{Personality}, coding specifically for if the paper utilizes a formal taxonomy of personality such as the Big Five model.
\item \textbf{Race}, regardless of the racial categorization framework used.
\item \textbf{Religion} or spirituality.
\item \textbf{Sexual orientation}, as well as romantic orientation. 
\item \textbf{Other group identity}, such as membership of a particular team or interest group.
\end{itemize}
